\documentclass [a4paper,12pt,fleqn]{article}
\usepackage {epsfig}
\def\ELAN {{\sf ELAN} }
\def\ELANns {{\sf ELAN}}
\def\IRIS {{\tt Iris} }
\def\IRISns {{\tt Iris}}
\def\Unix {{\sf Unix} }
\def\Unixns {{\sf Unix}}
\def\C {{\sf C} }
\def\Cns {{\sf C}}

% spacing
\newcommand {\idhalf}{\hspace*{.5em}}
\newcommand {\idone}{\hspace*{1em}}
\newcommand {\idtwo}{\hspace*{2em}}
\newcommand {\idten}{\hspace*{10em}}
\newcommand {\idfour}{\hspace*{4em}}
\newcommand {\idvert}{\vspace*{1em}}
% poetic citations
\newenvironment{mycite}%
{\begin{flushright} \begin{minipage}[t]{12in} \em \begin{tabbing}}%
{\end{tabbing} \end{minipage} \end{flushright}}
% list environments
\newcounter{letterctr}
\newenvironment{letterlist}%
{\begin{list} {\alph {letterctr})} {\usecounter {letterctr}}}%
{\end{list}}
\newcounter{capsctr}
\newenvironment{capslist}%
{\begin{list} {\Alph {capsctr})} {\usecounter {capsctr}}}%
{\end{list}}
\newcounter{ciphctr}
\newenvironment{ciphlist}%
{\begin{list} {\arabic {ciphctr})} {\usecounter {ciphctr}}}%
{\end{list}}
\newcounter{romanctr}
\newenvironment{romanlist}%
{\begin{list} {\arabic {romanctr})} {\usecounter {romanctr}}}%
{\end{list}}

\title {Description of the \ELAN Compiler\\
and the \ELAN Runtime System}
\author {M. Seutter}
\begin {document}
\maketitle
\section {Overview}
The \ELAN compiler (\verb+elancc+) translates programs written in \ELAN
to equivalent programs written in Assembler for a range of architectures.
Currently the Intel IA-32, Sparc, Alpha and VAX11 architectures are
supported. Upon succesfull compilation the (Gnu) assembler and linker
is called to assemble the program and link it against the \ELAN RunTime
System to create an executable:
\begin{center}
\epsfbox{flow2.eps}
\end{center}
\section {The \ELAN compiler}
\subsection {Basic structure of the \ELAN compiler}
The \ELAN compiler itself is written in \Cns. It translates \ELAN programs
in 5 passes, which communicate through common datastructures:
\vspace*{1ex}
\begin{center}
~\epsfbox{flow.eps}
\end{center}
\vspace*{1ex}
The first pass does lexical and syntactical analysis. The lexical analysis
recognizes program symbols, which are read by the syntactical analysis.
The syntactical analysis recognizes the structure of the program and
constructs a so-called abstract syntax tree (AST), representing this
structure. During this phase the USES clause of the program is checked
to decide whether certain user packets should be parsed as well (which
may necessitate the parsing of even more packets) and the order in which
they should be analyzed in the second pass. The first pass then finishes
by reading the precompiled standard prelude \verb+standard.ppe+, which
defines all predefined operations contained in the \ELAN compiler and
runtime environment. All these definitions are incorporated in an
abstract syntax tree, which is forwarded to the second pass of the compiler.

The second pass of the compiler is responsible for semantical analysis.
During semantical analysis, objects, procedures and operators are identified
and context conditions are checked. While the compiler is checking, it
decorates the abstract syntax tree with extra information, which eases
the next pass of the compiler.

The third pass of the compiler translates program constructs into
machine independent, yet close to actual hardware, abstract machine code.
This intermediate code (IMC) is based on quadruple notation and is
especially suited for machine independent optimization.

The fourth pass of the compiler is responsible for machine independent
code optimization. Currently, this pass is not yet implemented.

Finally, the fifth pass of the compiler translates the intermediate code
into Assembler for the target architecture. This translation is quite
straightforward. Typically, one instruction in the intermediate code
corresponds to two or three instructions in Assembler. Since no register
optimization is performed, the resulting code leaves much to be desired
in efficiency.

The previous figure also shows the module decomposition of the \ELAN compiler.
\subsection {Common modules}
\begin {list}{-}{}
\item [ {\tt elan\_ast.c} ]
This module describes the structure of the abstract syntax tree. It is
generated from the datastructure definition file {\tt elan\_ast.dcg}.
\item [ {\tt ast\_utils.c} ]
This module defines access routines to the abstract syntax tree as well
some usefull utilities.
\item [ {\tt elan\_imc.c} ]
This module describes the structure of the intermediate code (IMC).
One may view this intermediate code as a linear array of symbolic
instructions, some of which carry labels to allow them to be used as
targets of jumps.
It is generated from the datastructure definition file {\tt elan\_imc.dcg}.
\item [ {\tt imc\_utils.c} ]
This module defines access routines to the intermediate code as well
as a number of conversion routines.
\item [ {\tt decl\_tree.c} ]
This module defines the structure of the per packet tree of declarations.
These trees are constructed by the semantical analysis phase.
\end{list}
\subsection {Top level}
Three modules comprise the top level of the \ELAN compiler:
\begin {list}{-}{}
\item [ {\tt main.c} ]
This module defines the main program of the \ELAN compiler.
\item [ {\tt options.c} ]
This module parses the command line for compiler options and
the name of the source file.
\item [ {\tt backend.c} ]
This module is responsible for the spawning of the assembler and linker
after a succesfull compilation.
\end{list}
\subsection {Syntactical phase}
The first pass of the compiler is formed by the following three modules:
\begin {list}{-}{}
\item [ {\tt lexer.c} ]
The lexer is responsible for the lower level reading of \ELAN source files.
Layout and comments are discarded by the lexer, while tokenizing the input.
\item [ {\tt parser.c} ]
The parser recognizes the structure of the source program (and user packets)
and builds abstract syntax trees for these constructs. It is the largest
(and most documented) module of the \ELAN compiler.
\item [ {\tt contfree.c} ]
This module is responsible for combining the abstract syntax trees as
delivered by the parser of the source program and all user and system
packets the source program uses (directly or indirectly) into one
combined abstract syntax tree. Packets are then sorted according to their
use to define the order in which the semantical phase should analyze them.
\end{list}
\subsection {Semantical phase}
The second pass of the compiler consists only of seven modules:
\begin {list}{-}{}
\item [ {\tt contsens.c} ]
\item [ {\tt prechecker.c} ]
\item [ {\tt evaluate.c} ]
\item [ {\tt checker.c} ]
\item [ {\tt symbol\_table.c} ]
\item [ {\tt type\_table.c} ]
%This module checks if the source file (and the standard library) conform
%to the context conditions described in the \MINI language description by
%traversing and decorating the abstract syntax tree (or more accurately
%by breaking down the abstract syntax tree and reconstructing a decorated
%abstract syntax tree from the parts).
\item [ {\tt ident.c} ]
%This module checks prior to the traversal of the abstract syntax tree,
%whether it contains any (conflicting) double declarations. During the
%traversal, it tries to find for every application of a name a corresponding
%definition. Since \MINI allows generic (overloaded) procedure and operator
%declaration, both tasks are not trivial.
\end{list}
\subsection {Intermediate code generation}
Intermediate code generation is done by the module {\tt imc\_gen.c}.
It translates the source program and packets into an equivalent program
in the intermediate code by traversing the abstract syntax tree while
coding the program.
\subsection {Target code generation}
The last phase of the compiler consists of seven modules:
\begin{list}{-}{}
\item [ {\tt tgt\_gen.c} ]
This module does some initialization for code generation and then
calls the correct target code generator as determined from the options
and compiler configuration.
\item [ {\tt tgt\_gen\_machdep.c} ]
This module is responsible for recognition of certain target properties.
\item [ {\tt tgt\_gen\_common.c} ]
This module contains all functions that are common to all target code
generators.
\item [ {\tt tgt\_gen\_intel.c} ]
This module translates the intermediate code into Intel IA-32 Assembler.
\item [ {\tt tgt\_gen\_sparc.c} ]
This module translates the intermediate code into Sparc ($ \leq $V9) Assembler.
\item [ {\tt tgt\_gen\_alpha.c} ]
This module translates the intermediate code into Alpha Assembler.
\item [ {\tt tgt\_gen\_vax.c} ]
This module translates the intermediate code into VAX11 Assembler.
\end{list}
\section {Runtime model}
\subsection {Memory model and object representation}
The runtime model for \ELAN is based on the usual \C conventions of the
target machines (with some additions). We assume that memory consists of
words capable of holding one \verb+int+, which are either 4 or 8 bytes
long and correspondingly aligned. Addresses however are byte addresses.

The memory is partitioned in three parts, called the text, data and stack
segment, thus conforming to the usual \C and \Unix conventions. The text
segment holds the program code. The data segment holds {\tt TEXT} denotations
and the heap. This heap grows towards higher addresses. The stack is used
for procedure return addresses, actual parameters (when required by the \C
calling convention for the target machine (Note Sparc and Alpha)), local
variables and intermediate results and grows from high to low addresses. 

Integers, booleans, reals and addresses are represented by their actual values.
A {\tt TEXT} is represented by a pointer, pointing to a null terminated
string of characters, residing in the heap (when dynamically allocated)
or in the data segment (when pointing to a {\tt TEXT} denotation).
The null pointer is used to represent {\em uninitialized} {\tt TEXT} values.
Thus all objects in \ELAN use one or two machine words in the target
architecture (and can therefore be loaded in machine registers).
\subsection {Incremental garbage collection}
To support reuse of memory, an {\em incremental garbage collector} is
implemented for all objects in the heap. All objects in the heap are
therefore preceded by two {\em short} integers. The first of these shorts
denotes the number of pointers pointing to the object, the so-called
{\em reference count}. The other object denotes the {\em actual} object
length. To allow ease of reuse, all objects are allocated on double word
aligned addresses. Objects may therefore be padded with bytes, thus sometimes
causing the actual object length to be larger than the needed object length.

The basic idea of incremental garbage collection is that the reference count
of the object is incremented when a pointer starts to point to the object
and decremented when a pointer no longer points to the object. When the
reference count of an object drops to 0, it has become unreachable from
the running program and can therefore be reused. For historic reasons,
these two operations are called {\em attach} and {\em detach}. The
\verb+TEXT+ assignment in \ELAN can then be described by the following
three simple steps:
\begin{tabbing}
\hspace*{2em}\=Attach the source object \\
\>Detach the destination object\\
\>Copy source to destination (Only an address)
\end{tabbing}
The reader may figure out for herself why the attach operation {\em must}
be executed first.

{\tt TEXT} denotations have the same layout as heap objects. They differ
however in two aspects. They are allocated in the constant part of the
(read only) data segment. To indicate that they are constant and should
never be recycled by the garbage collector, their reference count equals -1.

The following three functions are available to interact with the garbage
collector:
\begin{list}{-}{}
\item [{\tt char~*rts\_malloc~(int~size)}] ~\\
This function allocates an object able to hold \verb+size+ bytes either
from the internal free list administration or from the heap. The initial
reference count of this object is (of course) 1. It is guaranteed that
the allocated object is double word aligned.
\item [{\tt char~*rts\_attach~(char~*ptr)}] ~\\
This function increases the reference count of the object, {\tt ptr} is
pointing to. When {\tt ptr} is a null pointer, {\tt rts\_attach} aborts.
When the original reference count was -1, {\tt rts\_attach} returns directly.
\item [{\tt void~rts\_detach~(char~**ptr)}] ~\\
This function decreases the reference count of the object, {\tt *ptr} is
pointing to while resetting {\tt *ptr}. When the reference count becomes 0,
the object is moved to the internal free list administration. When {\tt *ptr}
was a null pointer or was pointing to a {\tt TEXT} denotation,
{\tt rts\_detach} returns directly. 
\item [{\tt void~rts\_predetach~(char~**ptr)}] ~\\
More explanation needed.
\item [{\tt void~rts\_guard~(char~**ptr, int size)}] ~\\
Sets the guard bit. More explanation needed.
\item [{\tt void~rts\_init\_gc~()}] ~\\
This function initializes the free list administration.
\end{list}
\subsection {The runtime library {\tt liberts.a}}
The runtime library currently consists of the following 9 modules:
\begin{list}{-}{}
\item [{\tt rts\_init}] This module initializes the runtime system.
\item [{\tt rts\_error}] This module gives runtime errors.
\item [{\tt rts\_alloc}] The garbage collector. See above.
\item [{\tt rts\_ints}]
This module holds support routines for {\tt INT} operations.
\item [{\tt rts\_reals}]
This module holds support routines for {\tt REAL} operations.
\item [{\tt rts\_texts}]
This module holds support routines for {\tt TEXT} operations.
\item [{\tt rts\_files}]
This module holds support routines for {\tt FILE} operations.
\item [{\tt rts\_term}]
This module is responsible for the terminal interface.
\item [{\tt rts\_system}]
This module is responsible for the operating system interface.
\item [{\tt rts\_random}]
This module contains a random generator.
\end{list}
\end {document}
