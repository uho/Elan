\documentclass [a4paper,12pt,fleqn]{article}
\usepackage {../common/esprit,epsfig}
\def\ELAN {{\sf ELAN} }
\def\ELANns {{\sf ELAN}}
\def\IRIS {{\tt Iris} }
\def\IRISns {{\tt Iris}}
\def\Unix {{\sf Unix} }
\def\Unixns {{\sf Unix}}
\def\C {{\sf C} }
\def\Cns {{\sf C}}

% spacing
\newcommand {\idhalf}{\hspace*{.5em}}
\newcommand {\idone}{\hspace*{1em}}
\newcommand {\idtwo}{\hspace*{2em}}
\newcommand {\idten}{\hspace*{10em}}
\newcommand {\idfour}{\hspace*{4em}}
\newcommand {\idvert}{\vspace*{1em}}
% poetic citations
\newenvironment{mycite}%
{\begin{flushright} \begin{minipage}[t]{12in} \em \begin{tabbing}}%
{\end{tabbing} \end{minipage} \end{flushright}}
% list environments
\newcounter{letterctr}
\newenvironment{letterlist}%
{\begin{list} {\alph {letterctr})} {\usecounter {letterctr}}}%
{\end{list}}
\newcounter{capsctr}
\newenvironment{capslist}%
{\begin{list} {\Alph {capsctr})} {\usecounter {capsctr}}}%
{\end{list}}
\newcounter{ciphctr}
\newenvironment{ciphlist}%
{\begin{list} {\arabic {ciphctr})} {\usecounter {ciphctr}}}%
{\end{list}}
\newcounter{romanctr}
\newenvironment{romanlist}%
{\begin{list} {\arabic {romanctr})} {\usecounter {romanctr}}}%
{\end{list}}

\title {Description of the language \ELAN}
\author {M. Seutter}
\pagestyle {esprit}
\begin {document}
\maketitle
\section {Introduction}
\ELAN is to become a true superlanguage of {\sf ELAN}. The syntax of
\ELAN will be described by a context free grammar, augmented with
a number of metarules to improve readability of the grammar. For some
of the grammar rules simple examples are given. The context conditions
and the semantics for the syntactical constructs will be described
verbally if necessary.
\section {Abbreviations}
\subsubsection*{Syntax}
\begin {letterlist}
\item
\begin {verbatim}
NOTION option:
   NOTION;
   .
\end{verbatim}
\item
\begin{verbatim}
NOTION sequence:
   NOTION sequence, NOTION;
   NOTION.
\end{verbatim}
\item
\begin {verbatim}
NOTION list:
   NOTION list, comma token, NOTION;
   NOTION.
\end{verbatim}
\item
\begin{verbatim}
NOTION pack:
   open token, NOTION list, close token.
\end{verbatim}
\item
\begin{verbatim}
NOTION token:
   comments option, NOTION symbol.
\end{verbatim}
\end{letterlist}
\section {Programs and packets}
\subsubsection*{Syntax}
\begin {letterlist}
\item
\begin{verbatim}
elan program:
   packet sequence option, main packet.
\end{verbatim}
\item
\begin{verbatim}
packet:
   packet head, packet body, packet tail.
\end{verbatim}
\item
\begin{verbatim}
main packet:
   packet body.
\end{verbatim}
\item
\begin{verbatim}
packet head:
   packet token, packet name,
      export interface option, colon token.
\end{verbatim}
\item
\begin{verbatim}
export interface:
   defines token, export name list.
\end{verbatim}
\item
\begin{verbatim}
packet body:
   packet paragraph option, refinement sequence option.
\end{verbatim}
\item
\begin{verbatim}
packet paragraph:
   packet unit, semicolon token, packet paragraph option.
\end{verbatim}
\item
\begin{verbatim}
packet unit:
   basic declaration;
   closed declaration;
   import declaration.
\end{verbatim}
\item
\begin{verbatim}
packet tail:
   end packet token, packet name option, semicolon token.
\end{verbatim}
\item
\begin{verbatim}
import declaration:
   uses token, packet name list.
\end{verbatim}
\end{letterlist}
Packets form the basic compilation units of \ELANns.
\subsubsection*{Example}
\begin {verbatim}
fac:
   INT VAR i;
   put ("give number:\n");
   get (i);
   INT CONST i plus 1 :: i + 1;
   INT VAR fac :: 1;
   INT VAR counter1 :: 1;
   INT VAR counter2;
   INT VAR sum;
   WHILE counter1 < i plus 1
   REP
      sum := 0;
      counter2 := 0;
      WHILE counter2 < counter 1
      REP
         sum := sum + fac;
         counter2 := counter2 + 1
      ENDREP;
      counter1 := counter1 + 1
   ENDREP;
   put ("fac (");
   put (text (i));
   put (") = ");
   put (text (fac));
   put ("\n").
\end{verbatim}
\subsubsection*{Context conditions}
A program has type VOID (possibly after coercions).
Before parsing an \ELAN program, the \ELAN compiler reads the
standard prelude from the file \verb+standard.e+.
\section{Procedures and operators}
\subsubsection*{Syntax}
\begin{letterlist}
\item
\begin{verbatim}
procedure declaration:
   procedure head, routine body, procedure tail.
\end{verbatim}
\item
\begin{verbatim}
procedure head:
   result option, proc token, procedure name,
      formal parameter pack, colon token.
\end{verbatim}
\item
\begin{verbatim}
procedure tail:
   endproc token, procedure name option.
\end{verbatim}
\item
\begin{verbatim}
operator declaration:
   operator head, routine body, operator tail.
\end{verbatim}
\item
\begin{verbatim}
operator head:
   result, op token, operator name,
      formal parameter pack, colon token.
\end{verbatim}
\item
\begin{verbatim}
operator tail:
   endop token, operator name option.
\end{verbatim}
\item
\begin{verbatim}
routine body:
   internal token, text denotation;
   external token, text denotation;
   paragraph, routine body tail.
\end{verbatim}
\item
\begin{verbatim}
routine body tail:
   period token, refinement sequence;
   .
\end{verbatim}
\item
\begin{verbatim}
result:
   type declarer.
\end{verbatim}
\item
\begin{verbatim}
formal parameter:
   formal declarer, parameter name list.
\end{verbatim}
\end{letterlist}
\subsubsection*{Context conditions}
\begin{list}{-}{}
\item[a)]
A procedure declaration fixes the result type, name, argument type,
and body of the declared procedure.
\item[c)]
If present, the procedure name in the procedure tail must match
the procedure name in the procedure head.
\item[d)]
An operator declaration fixes the result type, name, number of arguments,
argument types and body of the declared operator.
\item[f)]
If present, the operator name in the operator tail must match
the operator name in the operator head.
\item[g)]
Internal routine bodies are directly translated by the \ELAN compiler,
as indicated by their internal name. External routine bodies are mapped
onto routines in the \ELAN runtime system, identified by their external name.
\end{list}
\subsubsection*{Examples}
\begin{list}{-}{}
\item[a)]
\begin{verbatim}
TEXT PROC text (INT CONST a):
   EXTERNAL "int_to_text"
ENDPROC;
\end{verbatim}
\item[d)] 
\begin{verbatim}
OP CAT (TEXT VAR a, TEXT b):
   a := a + b;
ENDOP CAT;
\end{verbatim}
\end{list}
\section{Refinements, paragraphs and units}
\subsubsection*{Syntax}
\begin{letterlist}
\item
\begin{verbatim}
refinement:
   refinement name, colon token,
      paragraph option, period token.
\end{verbatim}
\item
\begin{verbatim}
paragraph:
   paragraph, semicolon token, unit;
   unit.
\end{verbatim}
\item
\begin{verbatim}
unit:
   basic declaration;
   expression.
\end{verbatim}
\end{letterlist}
\subsubsection*{Context conditions}
\begin{list}{-}{}
\item[a)]
A refinement has the same type and access as the paragraph.
\item[b)]
A paragraph has the same type and access as the last unit in the
paragraph. All other units in the paragraph have type VOID.
An empty paragraph (paragraph option) has type VOID.
\item[c)]
A unit has the same type and access as the basic declaration or expression.
\end{list}
\subsubsection*{Semantics}
\begin{list}{-}{}
\item[ab)]
The declarations and expressions of a paragraph will be executed
sequentially from left to right. The result of a paragraph is
the result of the last unit.
\end{list}
\section {Declarations and declarers}
\subsubsection*{Syntax}
\begin{letterlist}
\item
\begin{verbatim}
closed declaration:
   procedure declaration;
   operator declaration;
   type declaration.
\end{verbatim}
\item
\begin{verbatim}
basic declaration:
   object declaration;
   synonym declaration.
\end{verbatim}
\end{letterlist}
\subsubsection*{Context conditions}
\begin{list}{-}{}
\item[a)]
A closed declaration has type VOID. It is not allowed to declare multiple
procedures or operators having the same name, number of arguments and same
argument types.
\end{list}
\subsection {Object declarations}
\subsubsection*{Syntax}
\begin{letterlist}
\item
\begin{verbatim}
object declaration:
   object declarer, object association list.
\end{verbatim}
\item
\begin{verbatim}
object association:
   object name, object initialization option.
\end{verbatim}
\item
\begin{verbatim}
object initialization:
   initial token, expression.
\end{verbatim}
\end{letterlist}
\subsubsection*{Context conditions}
\begin{list}{-}{}
\item[a)]
An object declaration has type VOID.
\item[b)]
A variable may have an initialization. A constant must have an initialization.
\end{list}
\subsection {Synonym declarations}
\subsubsection*{Syntax}
\begin{letterlist}
\item
\begin{verbatim}
synonym declaration:
   let token, synonym association list.
\end{verbatim}
\item
\begin{verbatim}
synonym assocation:
   synonym value association;
   synonym type association.
\end{verbatim}
\item
\begin{verbatim}
synonym value assocation:
   synonym value name, equal token, expression.
\end{verbatim}
\item
\begin{verbatim}
synonym type association:
   synonym type name, equal token, type declarer.
\end{verbatim}
\end{letterlist}
\subsection {Type declarations}
\subsubsection*{Syntax}
\begin{letterlist}
\item
\begin{verbatim}
type declaration:
   type token, type association list.
\end{verbatim}
\item
\begin{verbatim}
type association:
   type name, equal token, type declarer.
\end{verbatim}
\end{letterlist}
\subsection {Declarers}
\subsubsection*{Syntax}
\begin{letterlist}
\item
\begin{verbatim}
formal declarer:
   object declarer.
\end{verbatim}
\item
\begin{verbatim}
object declarer:
   type declarer, access declarer option.
\end{verbatim}
\item
\begin{verbatim}
type declarer:
   elementary type declarer;
   composed type declarer.
\end{verbatim}
\item
\begin{verbatim}
elementary type declarer:
   concrete type declarer;
   abstract type declarer.
\end{verbatim}
\item
\begin{verbatim}
concrete type declarer:
   int token;
   bool token;
   real token;
   text token.
\end{verbatim}
\item
\begin{verbatim}
abstract type declarer:
   type name.
\end{verbatim}
\item
\begin{verbatim}
composed type declarer:
   row declarer;
   struct declarer;
   union declarer.
\end{verbatim}
\item
\begin{verbatim}
row declarer:
   row token, cardinality, type declarer.
\end{verbatim}
\item
\begin{verbatim}
cardinality:
   denoter;
   synonym value name.
\end{verbatim}
\item
\begin{verbatim}
struct declarer:
   struct token, field specification pack.
\end{verbatim}
\item
\begin{verbatim}
field specification:
   type declarer, field name list.
\end{verbatim}
\item
\begin{verbatim}
access declarer:
   const token;
   var token.
\end{verbatim}
\end{letterlist}
\section {Expressions}
\subsubsection*{Syntax}
\begin{letterlist}
\item
\begin{verbatim}
expression:
   assignment;
   priority ii formula.
\end{verbatim}
\end{letterlist}
\subsubsection*{Context conditions}
\begin{list}{-}{}
\item[a)]
The expression has the same type and access as the assignment or
priority ii formula.
\end{list}
\subsection{Assignments}
\subsubsection*{Syntax}
\begin{letterlist}
\item
\begin{verbatim}
assignment:
   destination, becomes token, source.
\end{verbatim}
\item
\begin{verbatim}
destination:
   priority ii formula.
\end{verbatim}
\item
\begin{verbatim}
source:
   priority ii formula.
\end{verbatim}
\end{letterlist}
Note that syntactically, the assignment acts as a priority i formula.
\subsubsection*{Context conditions}
\begin{list}{-}{}
\item[a)]
The assignment has the type VOID. The destination and source must
have the same type.
\item[b)]
The destination has access VAR.
\item[c)]
The source has access CONST.
\end{list}
\subsubsection*{Semantics}
\begin{list}{-}{}
\item[a)]
The left and right hand side of the assignment will be evaluated.
Afterwards the value calculated for the right hand side will be
assigned to the variable of the left hand side.
\end{list}
\subsection {Formulas}
\subsubsection*{Syntax}
\begin{letterlist}
\item
\begin{verbatim}
priority ii formula:
   priority ii formula, priority ii operator, priority iii formula;
   priority iii formula.
\end{verbatim}
\item
\begin{verbatim}
priority ii operator:
   capital name.
\end{verbatim}
\item
\begin{verbatim}
priority iii formula:
   priority iii formula, priority iii operator, priority iv formula;
   priority iv formula.
\end{verbatim}
\item
\begin{verbatim}
priority iii operator:
   or token.
\end{verbatim}
\item
\begin{verbatim}
priority iv formula:
   priority iv formula, priority iv operator, priority v formula;
   priority v formula.
\end{verbatim}
\item
\begin{verbatim}
priority iv operator:
   and token.
\end{verbatim}
\item
\begin{verbatim}
priority v formula:
   priority v formula, priority v operator, priority vi formula;
   priority vi formula.
\end{verbatim}
\item
\begin{verbatim}
priority v operator:
   equal token;
   not equal token.
\end{verbatim}
\item
\begin{verbatim}
priority vi formula:
   priority vi formula, priority vi operator, priority vii formula;
   priority vii formula.
\end{verbatim}
\item
\begin{verbatim}
priority vi operator:
   less than token;
   less equal token;
   greater than token;
   greater equal token.
\end{verbatim}
\item
\begin{verbatim}
priority vii formula:
   priority vii formula, priority vii operator, priority viii formula;
   priority viii formula.
\end{verbatim}
\item
\begin{verbatim}
priority vii operator:
   plus token;
   minus token.
\end{verbatim}
\item
\begin{verbatim}
priority viii formula:
   priority viii formula, priority viii operator, priority ix formula;
   priority ix formula.
\end{verbatim}
\item
\begin{verbatim}
priority viii operator:
   asterix token;
   div token;
   mod token.
\end{verbatim}
\item
\begin{verbatim}
priority ix formula:
   priority ix formula, priority ix operator, monadic formula;
   monadic formula.
\end{verbatim}
\item
\begin{verbatim}
priority ix operator:
   obelix token.
\end{verbatim}
\item
\begin{verbatim}
monadic formula:
   monadic operator, monadic formula;
   primary.
\end{verbatim}
\item
\begin{verbatim}
monadic operator:
   capital name;
   plus token;
   minus token;
   not token.
\end{verbatim}
\end{letterlist}
\subsubsection*{Context conditions}
\begin{list}{-}{}
\item[a,c,e,g,i,k,m,o)]
The types of the operands of the dyadic operator are determined.
Using these types and the name of the dyadic operator, the corresponding
operator definition is identified among the predefined operators.
The type of the formula is the same as the result type of the
identified operator; the access is CONST. The operands of the dyadic
operator must have the same type and access as the corresponding formal
parameter of the identified operator.
\item[q)]
The type of the operand of the monadic operator is determined.
Using this type and the name of the monadic operator, the corresponding
operator definition is identified among the predefined operators.
The type of the monadic formula is the same as the result type of
the identified operator; the access is CONST. The operand of the
monadic operator must have the same type and access as the corresponding
formal parameter of the identified operator. 
\end{list}
\subsubsection*{Semantics}
\begin{list}{-}{}
\item[a,b)]
The corresponding operator will be called after the evaluation of
the operands. The result of the application of operators should be
as expected by the conventions of analysis.
\end{list}
\section {Primaries}
\subsubsection*{Syntax}
\begin{letterlist}
\item
\begin{verbatim}
primary:
   conditional choice;
   numerical choice;
   repetition;
   terminator;
   closed clause;
   display;
   subscription;
   selection;
   abstractor;
   concretizer;
   nil;
   call;
   denoter;
   identifier application.
\end{verbatim}
\item
\begin{verbatim}
closed clause:
   open token, paragraph, close token.
\end{verbatim}
\item
\begin{verbatim}
display:
   sub token, expression list, bus token.
\end{verbatim}
\item
\begin{verbatim}
subscription:
   primary, sub token, expression, bus token.
\end{verbatim}
\item
\begin{verbatim}
selection:
   primary, period token, field name.
\end{verbatim}
\item
\begin{verbatim}
abstractor:
   type name, colon token, primary.
\end{verbatim}
\item
\begin{verbatim}
concretizer:
   concr token, primary.
\end{verbatim}
\item
\begin{verbatim}
nil:
   nil token.
\end{verbatim}
\item
\begin{verbatim}
call:
   procedure name, actual parameter list pack.
\end{verbatim}
\item
\begin{verbatim}
actual parameter:
   expression.
\end{verbatim}
\item
\begin{verbatim}
identifier application:
   refinement name;
   procedure name;
   constant name;
   variable name.
\end{verbatim}
\end{letterlist}
\subsubsection*{Context conditions}
\begin{list}{-}{}
\item [a)]
The primary has the same type and access as the conditional clause,
repetition, call, denoter, enclosed expression or identifier application.
\item [c)]
The type of the actual parameter of the call is determined.
Using this type and the procedure name, the corresponding
procedure definition is identified among the defined procedures.
The type of the call is the same as the result type of
the identified procedure; the access is CONST. The actual parameter
of the call must have the same type and access as the corresponding
formal parameter of the identified procedure. 
\item [e)]
The type and access of an identifier application is the type
and access of the object whose declaration defined the identifier.
However if the access demanded by the context is CONST and the
identified object is a variable, the access of the identifier
application is coerced to CONST.
\end{list}
\subsubsection*{Semantics}
\begin{list}{-}{}
\item[b)]
The corresponding procedure will be called after the evaluation of
the argument. 
\item[d)]
An identifier application yields the identified object. Needs more
elaboration
\end{list}
\section {Control constructs}
\subsection {The conditional choice}
\subsubsection*{Syntax}
\begin{letterlist}
\item
\begin{verbatim}
conditional choice:
   if token, condition, then part, else part option,
      end if token.
\end{verbatim}
\item
\begin{verbatim}
condition:
   expression.
\end{verbatim}
\item
\begin{verbatim}
then part:
   then token, paragraph.
\end{verbatim}
\item
\begin{verbatim}
else part:
   else token, paragraph;
   elif token, condition, then part, else part option.
\end{verbatim}
\end{letterlist}
\subsubsection*{Context conditions}
\begin{list}{-}{}
\item[a)]
The conditional choice has the same type and access as the then part
and else part option. If the else part option is empty, it has type VOID.
\item[b)]
The condition has type BOOL and access CONST.
\item[c)]
The then part has the same type and access as the paragraph.
\item[d)]
The else part has the same type and access as the paragraph
or the then part and else part option. If the else part option is
empty, it has type VOID.
\end{list}
\subsubsection*{Semantics}
\begin{list}{-}{}
\item[a)]
In a conditional choice, the condition is evaluated first. If the condition
yields the value TRUE, the then part will be executed. If it yields the
value FALSE, the else part will be executed if present.
\item[d)]
In an else part conforming to the second alternative, the condition is
evaluated first. If the condition yields the value TRUE, the then part will
be executed. If it yields the value FALSE, the else part will be executed if
present.
\end{list}
\subsection {The numerical choice}
\subsubsection*{Syntax}
\begin{letterlist}
\item
\begin{verbatim}
numerical choice:
   select token, expression, of token, case part sequence,
      otherwise part option, end select token.
\end{verbatim}
\item
\begin{verbatim}
case part:
   case token, case label list, colon token, paragraph option.
\end{verbatim}
\item
\begin{verbatim}
case label:
   denoter;
   synonym value name.
\end{verbatim}
\item
\begin{verbatim}
otherwise part:
   paragraph option.
\end{verbatim}
\end{letterlist}
\subsubsection*{Context conditions}
\begin{list}{-}{}
\item[a)]
The numerical choice has the same type and access as the case part sequence
and the otherwise part option. If the otherwise part option is emprt, it has
type VOID. The expression has type INT and access CONST. All parts of the
case part sequence have the same type and access.
\item[b)] 
The case part has the same type and access as the paragraph.
\item[c)]
The case label must have type INT. Its integer value should be statically known.
\item[d)]
The otherwise part has the same type and access as the paragraph.
\end{list}
\subsubsection*{Semantics}
\begin{list}{-}{}
\item[a)]
In a numerical choice, the expression is evaluated first. If the expression
yields a value which is equal to the value of one of the case labels, the
corresponding paragraph will be executed. If it does not match any case label,
the otherwise part will be executed if present.
\end{list}
\subsection {The repetition}
\subsubsection*{Syntax}
\begin{letterlist}
\item
\begin{verbatim}
repetition:
   for part option, from part option,
      direction part option, while part option,
      rep token, paragraph, until part option,
      endrep token.
\end{verbatim}
\item
\begin{verbatim}
for part:
   for token, loop variable name.
\end{verbatim}
\item
\begin{verbatim}
loop variable name:
   small name.
\end{verbatim}
\item
\begin{verbatim}
from part:
   from token, expression.
\end{verbatim}
\item
\begin{verbatim}
direction part:
   upto token, expresion;
   downto token, expression.
\end{verbatim}
\item
\begin{verbatim}
while part:
   while token, condition.
\end{verbatim}
\item
\begin{verbatim}
until part:
   until token, condition.
\end{verbatim}
\end{letterlist}
\subsubsection*{Context conditions}
\begin{list}{-}{}
\item[a)]
The repetition always has type void.
\item[b)]
The loop variable must be declared.
It must have type INT and access VAR.
\item[d)]
The expression must have type INT and access CONST.
\item[e)]
The expressions must have type INT and access CONST.
\item[f)]
The condition must have type BOOL and access CONST.
\item[g)]
The condition must have type BOOL and access CONST.
\end{list}
\subsubsection*{Semantics}
If the for part is absent, an anonymous variable of type INT and access VAR
is used as the loop variable. Initially the from part, if present, is
evaluated and assigned to the loop variable. If the from part is absent
the value 1 is assigned to the loop variable. Next the direction part,
if present, is evaluated.

The evaluation of a repetition then consists of continuously iterating the
following steps:
\begin{list}{-}{}
\item
If the direction part is present as its first alternative, the value of the
loop variable is compared with the value evaluated for the direction part.
If it is greater, the evaluation of the repetition is finished.

If the direction part is present as its second alternative, the value of the
loop variable is compared with the value evaluated for the direction part.
If it is less, the evaluation of the repetition is finished.
\item
If present, the while part, is evaluated. If it yields the value FALSE,
the evaluation of the repetition is finished. If it yields the value TRUE or
if the while part is absent, the evaluation of the repetition continues.
\item
The paragraph is evaluated.
\item
If present, the until part, is evaluated. If it yields the value TRUE,
the evaluation of the repetition is finished. If it yields the value FALSE or
if the until part is absent, the evaluation of the repetition continues.
\item
If the direction part is absent or if the direction part is present as its
first alternative, the value of the loop variable is incremented.

If the direction part is present as its second alternative, the value of the
loop variable is decremented.
\end{list}
\subsection {The terminator}
\subsubsection*{Syntax}
\begin{letterlist}
\item
\begin{verbatim}
terminator:
   leave token, algorithm name, premature result option.
\end{verbatim}
\item
\begin{verbatim}
premature result:
   with token, expression.
\end{verbatim}
\end{letterlist}
\section{Names}
\subsubsection*{Syntax}
\begin{letterlist}
\item
\begin{verbatim}
export name:
   constant name;
   procedure name;
   operator name;
   type name.
\end{verbatim}
\item
\begin{verbatim}
packet name:
   small name.
\end{verbatim}
\item
\begin{verbatim}
refinement name:
   small name.
\end{verbatim}
\item
\begin{verbatim}
constant name:
   small name.
\end{verbatim}
\item
\begin{verbatim}
variable name:
   small name.
\end{verbatim}
\item
\begin{verbatim}
parameter name:
   small name.
\end{verbatim}
\item
\begin{verbatim}
procedure name:
   small name.
\end{verbatim}
\item
\begin{verbatim}
field name:
   small name.
\end{verbatim}
\item
\begin{verbatim}
operator name:
   capital name;
   special name.
\end{verbatim}
\item
\begin{verbatim}
type name:
   capital name.
\end{verbatim}
\item
\begin{verbatim}
capital name:
   comments option, [A-Z][A-Z0-9]*,
   { A capital name may not be one of the reserved
     words, named in the section 10 }.
\end{verbatim}
\item
\begin{verbatim}
small name:
   small name, comments option, [a-z0-9]+;
   comments option, [a-z][a-z0-9]*.
\end{verbatim}
\item
\begin{verbatim}
special name:
   or token;
   and token;
   equal token;
   not equal token;
   less token;
   less equal token;
   greater token;
   greater equal token;
   plus token;
   minus token;
   asterix token;
   divides token;
   div token;
   mod token;
   obelix token;
   not token.
\end{verbatim}
\end{letterlist}
\section {Denoters}
\subsubsection*{Syntax}
\begin{letterlist}
\item
\begin{verbatim}
denoter:
   comments option, denotation.
\end{verbatim}
\item
\begin{verbatim}
denotation:
   integer denotation;
   boolean denotation;
   real denotation;
   text denotation.
\end{verbatim}
\item
\begin{verbatim}
integer denotation:
   digit sequence.
\end{verbatim}
\item
\begin{verbatim}
boolean denotation:
   true symbol;
   false symbol.
\end{verbatim}
\item
\begin{verbatim}
real denotation:
   digit sequence, period symbol,
      digit sequence, exponent option.
\end{verbatim}
\item
\begin{verbatim}
exponent:
   e symbol, sign option, digit sequence.
\end{verbatim}
\item
\begin{verbatim}
text denotation:
   quote symbol, text item sequence option, quote symbol.
\end{verbatim}
\item
\begin{verbatim}
digit:
   [0-9].
\end{verbatim}
\item
\begin{verbatim}
text item:
   ([^\\\n\"]|\\.).
\end{verbatim}
\end{letterlist}
\subsubsection*{Context conditions}
\begin{list}{-}{}
\item[a)]
The type of the denoter is the same as the type of the
denotation. The access of a denoter is CONST.
\item[b)]
The type of a denotation is INT if it is an integer denotation,
REAL if it is a real denotation, BOOL if it is a boolean denotation and
TEXT if it is a text denotation.
\end{list}
\section {Symbols}
The following table describes all symbols used in the \ELAN language.
Layout inside the symbols is not allowed.
\begin{center}
\begin{tabular}{lll}
uses symbol & & \verb+USES+\\
end symbol & & \verb+END+\\
proc symbol & & \verb+PROC+\\
endproc symbol & & \verb+ENDPROC+\\
op symbol & & \verb+OP+\\
endop symbol & & \verb+ENDOP+\\
let symbol & & \verb+LET+\\
type symbol & & \verb+TYPE+\\
int symbol & & \verb+INT+\\ 
real symbol & & \verb+REAL+\\
bool symbol & & \verb+BOOL+\\
text symbol & & \verb+TEXT+\\
row symbol & & \verb+ROW+\\
array symbol & & \verb+ARRAY+\\
struct symbol & & \verb+STRUCT+\\
const symbol & & \verb+CONST+\\
var symbol & & \verb+VAR+\\
if symbol & & \verb+IF+\\
then symbol & & \verb+THEN+\\
elif symbol & & \verb+ELIF+\\
else symbol & & \verb+ELSE+\\
endif symbol & & \verb+ENDIF+\\
for symbol & & \verb+FOR+\\
from symbol & & \verb+FROM+\\
upto symbol & & \verb+UPTO+\\
downto symbol & & \verb+DOWNTO+\\
while symbol & & \verb+WHILE+\\ 
until symbol & & \verb+UNTIL+\\
rep symbol & & \verb+REP+\\
endrep symbol & & \verb+ENDREP+\\
select symbol & & \verb+SELECT+\\
of symbol & & \verb+OF+\\
case symbol & & \verb+CASE+\\
otherwise symbol & & \verb+OTHERWISE+\\
endselect symbol & & \verb+ENDSELECT+\\
concr symbol & & \verb+CONCR+\\
leave symbol & & \verb+LEAVE+\\
with symbol & & \verb+WITH+\\
\hspace*{8em} & \hspace*{2em} & \hspace*{8em}
\end {tabular}
\end {center}
\begin{center}
\begin{tabular}{lll}
colon symbol & & \verb+:+\\
initial symbol & & \verb+::+\\
becomes symbol & & \verb+:=+\\
semicolon symbol & & \verb+;+\\
period symbol & & \verb+.+\\
comma symbol & & \verb+,+\\
open symbol & & \verb+(+\\
close symbol & & \verb+)+\\
sub symbol & & \verb+[+\\
bus symbol & & \verb+]+\\
plus symbol & & \verb-+-\\
minus symbol & & \verb+-+\\
asterix symbol & & \verb+*+\\
divides symbol & & \verb+/+\\
obelix symbol & & \verb+**+\\
equal symbol & & \verb+=+\\
not equal symbol & & \verb+<>+\\
less than symbol & & \verb+<+\\
less equal symbol & & \verb+<=+\\
greater than symbol & & \verb+>+\\
greater equal symbol & & \verb+>=+\\
or symbol & & \verb+OR+\\
and symbol & & \verb+AND+\\
div symbol & & \verb+DIV+\\
mod symbol & & \verb+MOD+\\
not symbol & & \verb+NOT+\\
true symbol & & \verb+TRUE+\\
false symbol & & \verb+FALSE+\\
e symbol & & \verb+E+\\
quote symbol & & \verb+"+\\
comment open symbol & & \verb+{+\\
comment close symbol & & \verb+}+\\
\hspace*{8em} & \hspace*{2em} & \hspace*{8em}
\end {tabular}
\end {center}
The following table describes all additional symbols allowed in
packets:
\begin{center}
\begin{tabular}{lll}
packet symbol & & \verb+PACKET+\\
endpacket symbol & & \verb+ENDPACKET+\\
defines symbol & & \verb+DEFINES+\\
all symbol & & \verb+ALL+\\
internal symbol & & \verb+INTERNAL+\\
external symbol & & \verb+EXTERNAL+\\
\hspace*{8em} & \hspace*{2em} & \hspace*{8em}
\end {tabular}
\end {center}
\section {Comments and layout}
\subsubsection*{Syntax}
\begin {letterlist}
\item
\begin{verbatim}
comments:
   comments option, comment, layout;
   layout.
\end{verbatim}
\item
\begin{verbatim}
comment:
   comment open symbol, comment chars,
      comment close symbol.
\end{verbatim}
\item
\begin{verbatim}
comment chars:
   [^{}]*.
\end{verbatim}
\item
\begin{verbatim}
layout:
   [ \t\n\r]*.
\end{verbatim}
\end {letterlist}
\subsubsection*{Example}
\begin{verbatim}
   { This is a comment }
\end{verbatim}
\subsubsection*{Semantics}
Comments are meaningless for the execution of the program. They
only serve as documentation for human readers of programs.
\section{Coercions}
There are two coercions in \ELANns, namely:
\begin{ciphlist}
\item
consting: when demanded by the context the access of a variable
application is forced to CONST. This yields the value of the variable.
\item
voiding: may be applied to any expression. The value of the expression
is forgotten. The result has the type VOID.
\end{ciphlist}
\end {document}
